\documentclass[12pt]{article}
\usepackage[utf8]{inputenc}
\usepackage{hyperref}
\usepackage{listings}
\usepackage{color}

\definecolor{codegray}{gray}{0.95}
\lstset{
  backgroundcolor=\color{codegray},
  basicstyle=\ttfamily\footnotesize,
  breaklines=true,
  frame=single,
  columns=fullflexible
}

\begin{document}

\section*{Title}




\section*{Authors}





\section*{Abstract}





\section*{Introduction}





\subsection*{Development of the protocol}





\subsection*{Overview of the procedure}





\subsection*{Applications of the method}





\subsection*{Comparison with other methods}





\subsection*{Experimental Design}





\subsubsection*{Reference data (Step 1)}

\paragraph*{A.  Reference Data}


We follow the \href{https://github.com/broadinstitute/gtex-pipeline/blob/master/TOPMed\\_RNAseq\\_pipeline.md}{TOPMed workflow from Broad} for the aquisition and preparation of reference files for use throughout our pipeline. Our processed reference data may be found \href{https://www.synapse.org/#!Synapse:syn36416587}{in this folder on Synapse}. We have included the PDF document compiled by Data Standardization Working Group in the \href{https://www.synapse.org/#!Synapse:syn36416587}{on Synapse} as well as on \href{https://www.niagads.org/adsp/content/adspgcadgenomeresources-v2pdf}{ADSP Dashboard}. It contains the reference data to use for the project.




During fine mapping we often use cis window (up & downstream 1M base of te start of gene) as the fine mapping region. However, this method lacks justification in biological level, so we will use topologically associated domains (TAD) to create a list of fine mapping regions.



We use generalized TAD region lists created from the merging of TAD data from hippocampus and cortex brain tissues obtained from \href{https://doi.org/10.1016%2Fj.ajhg.2021.01.001}{[cf. McArthur et al (2021)} and \href{https://doi.org/10.1016%2Fj.celrep.2016.10.061}{[cf. Schmitt et al (2017)}. Variants within boundaries may also have an effect on gene expresion, so we explore how to extend TAD regions to the adjacent boundaries (TADB), in an attempt to not leave out any possible causal variants.



However, we are trying to be conservative, so we do not want to lose information gained from the use of 1Mb cis windows. Therefore, we want to extend those regions. Here we extend both TADB and TADB-enhanced cis windows since each has a different goal.



To build a list of extended TADB, we start with generalized TADB. We find all genes within each TADB and get their start and end positions extended by 1Mb cis windows. We then take the outmost boundary of all the !M cis windows and this TADB as the boundary of the extended TADB. We end with 1,381 TADBs which may then be used as functional units of epigenetic analysis.



To build a list of TADB-enhanced cis windows, we start with the cis window of each gene. We take the outermost boundary of the generalized TADB the gene is in and the 1Mb cis window of the gene as the boundary of the TADB-enhanced cis window. If one gene is in two generalized TADBs, we take the outermost of both the TADB and cis window. We end with a cis window for each gene.




Our LD reference panel is generated from ADSP GCAD non-Hispanic white samples. Missing variants are mean imputed before correlations are calculated between variants. The cyvcf2 package is used to calculate dosage while applying a minor allele frequency threshold of 0.05%, a minor allele count threshold of 5, and a missingness threshold of 5%.

\subsubsection*{Molecular Phenotypes (Step 2)}

\paragraph*{A.  RNA-seq expression}


Our pipeline follows the GTEx pipeline from the \href{https://gtexportal.org/home/aboutGTEx#staticTextAnalysisMethods}{GTEx}/\href{https://github.com/broadinstitute/gtex-pipeline/blob/master/TOPMed\\_RNAseq\\_pipeline.md}{TOPMed} project for the quantification of expression from RNA-seq data. Either paired end or single end fastq.gz files may be used as the initial input. Different read \href{https://rnabio.org/module-09-appendix/0009/12/01/StrandSettings/}{strandedness options} are possible, including rf, fr, unstranded or strand\_missing, based on library types from Signal et al \href{https://doi.org/10.1186/s12859-022-04572-7}{[cf. Signal et al (2022)}]. Strand detection steps are included in this pipeline to automaticaly detect the strand of the input fastq file by levradging the gene count table ouptut from \href{https://physiology.med.cornell.edu/faculty/skrabanek/lab/angsd/lecture\\_notes/STARmanual.pdf}{STAR}. Read length data is required and is set to a default value of 100 for reads of zero length.



We recommend using fastqc to quality control reads, followed by the removal of adaptors with fastp if necessary. After quality control has been conducted, STAR may be used to align reads to the reference genome. We combine this mapping step with an additionaly quality control step using Picard to mark duplicate reads and collect other metrics.



Gene-level RNA expression is called with RNA-SeQC v2.4.2 using a reference gtf that has been collapsed to contain genes instead of gene transcripts \href{https://doi.org/10.1093/bioinformatics/bts196}{[cf. DeLuca et al., Bioinformatics, 2012}]. Reads are only included if they are uniquely mapped (mapping quality of 255 for STAR BAMs), if they are aligned in proper pairs, if the read alignment distance was less than or equal to six (i.e. alignments must not contain more than six non-reference bases), and if they are fully contained within exon boundaries. Reads overlapping introns are not included. Exon level read counts are also called by RNA-SeQC. If a read overlaps multiple exons, then a fractional value equal to the portion of the read contained within the exon is allotted. We call transcript-level RNA expression using RSEM v1.3.0 with a transcript reference gtf.

Quality control of the TPM matrices follows methods outlined by \href{https://gtexportal.org/home/aboutGTEx#staticTextAnalysisMethods}{GTEx V8}. First, genes are removed from the matrices if over 20% of samples have have a TPM expression level of 10% or less. Sample level filtering includes three checks to detect sample outliers. Samples are only removed if they are marked as outliers in all three checks.



The first looks at Relative Log Expression (RLE). It is assumed that most gene expression values in a sample should be near a mean value and only a few genes are differentially expressed. To calculate RLE, for each gene *i*, calculate its median in *N* samples as *Medi*. Then for each sample *j* and expression value *eij*, count the difference between *eij* and *Medi* (*deij = eij-Medi*). Then create boxplots for each sample based on *deij* and sort by interquartile range. Samples with larger interquartile ranges are more likely to be outliers.



The second quality control check looks at heirarchical clustering of samples. Samples are expected to have short distances between others and therefore should cluster homogeneously, so distant samples are expected to be outliers. Distance is calculated as 1-spearman correlation for heirarchical clustering. The top 100 genes sorted by variance are used to calculate Mahalanobis distance. Chi2 p-values are then calculated based on Mahalanobis distance. Clusters with 60% or more samples with Bonferroni corrected p-values less than 0.05 are marked as outliers.



The third and last quality control check looks at D-statistics which represent the average correlation between a sample's expression and other sampels. Samples with low D-statistics are likely to be outliers.









The normalization step follows steps used by the GTeX pipeline. Genes are first filtered to keep genes where TPM is greater than 10% in at least 20% of the samples. They are also kept if read counts is greater than 6 in at least 20% of the samples. The filtered data is then normalized using the Trimmed Mean of M-value (TMM) method.




\paragraph*{B.  Alternative splicing from RNA-seq data}


Our pipeline calls alternative splicing events from RNA-seq data using leafcutter and psichomics to call the RNA-seq data from `fastq.gz` data which has been mapped to a reference genome using STAR with the wasp option. It implements the GTEx pipeline for GTEx/TOPMed project. Please refer to \href{https://github.com/broadinstitute/gtex-pipeline/blob/master/TOPMed\\_RNAseq\\_pipeline.md}{this page} for detail. The choice of pipeline modules in this project is supported by internal (unpublished) benchmarks from GTEx group.



We use two different tools to quantify the many types of splicing events which are outlined in \href{https://doi.org/10.1038/nature07509}{[cf. Wang et al (2008)}] and \href{https://doi.org/10.1016/j.ajhg.2017.11.002}{[cf. Park et al (2018)}]. The first, leafcutter, quantifies the usage of alternatively excised introns. This collectively captures skipped exons, 5’ and 3’ alternative splice site usage and other complex events \href{https://doi.org/10.1038/s41588-017-0004-9}{[cf. Li et al 2018}]. This method was previously applied to ROSMAP data as part of the Brain xQTL version 2.0.  The second, psichomics, quantifies specific splicing events \href{https://doi.org/10.1093/nar/gky888}{[cf. Agostinho et al. 2019}].


Quality control and normalization are performed on output from the leafcutter and psichomics tools. The raw output data is first converted to bed format. Quality control involves the removal of features with high missingness across samples (default 40%) and any remaining missing values are retained. Then introns with less than a minimal variation (default of 0.001) are removed from the data. Quantile-Quantile normalization is performed on the quality controlled data. Imputation of missing values is done separately afterwards.

\subsubsection*{Data Pre-processing (Step 3)}

\paragraph*{A.  Genotype data preprocessing}


A major challenge in biomedical research is the quality control (QC) of sequencing data. False positive variant calls can hinder the ability to detect disease associated variants or introduce spurious associations, therefore the need for a rigorous QC. Our pipeline focuses on QC after the variant calling stage and requires project Variant Calling Format (pVCF) as input files. We have defined default thresholds for genotype and variant-level hard filtering based on recommendations from the UK Biobank team and a thorough review of the literature \href{https://doi.org/10.1186/1471-2105-15-125}{[cf. Carson et al. BMC Bioinformatics (2014)},\href{https://doi.org/10.1038/nature19057}{cf. Lek et al. Nature (2016)},\href{https://doi.org/10.1038/s41588-021-00885-0}{cf. Szustakowski et al. Nature Genetics (2021)}]. Bcftools is used in our QC steps. We first handle multi-allelic sites by splitting them into bi-allelic records. We include an optional workflow to keep only bi-allelic sites in the data. Variants are then annotated based on dbSNP data. Genotypes are kept if they have a Genotype Depth (DP) >= 10 and a Genotype Quality (GQ) >= 20. Variants are included if at least one sample has an allelic balance (AB) >= 0.15 for Single Nucleotide Variants (SNVs) and AB>=0.2 for indels, variant missigness is below 20% and Hardy-Weinberg Equilibrium p-value is > 1e-08. Allele balance is calculated for heterozygotes as the number of bases supporting the least-represented allele over the total number of base observations. Output summary statistics, such as transistion/transversion ratios (TS/TV ratio) are calculated to determine the effectiveness of QC.

We include steps for the formatting of genotype files. This includes the conversion between VCF and PLINK formats, the splitting of data (by specified input, chromosomes or genes) and the merging of data (by specified input, or by chromosomes).

\paragraph*{B.  Phenotype data preprocessing}


We use a gene coordinate annotation pipeline based on \href{https://github.com/broadinstitute/gtex-pipeline/blob/master/qtl/src/eqtl\\_prepare\\_expression.py}{`pyqtl`, as demonstrated here}. This adds genomic coordinate annotations to gene-level molecular phenotype files generated in `gct` format and converts them to `bed` format for downstreams analysis.


Empirical Bayes Matrix Factorization (EBMF) is the primary imputation method we use to impute molecular phenotype data \href{https://doi.org/10.1101/2023.11.29.23299181}{[cf. Qi et al., medRxiv,  2023}]. We also provide a collection of other methods to impute missing omics data values including missing forest, XGboost, k-nearest neighbors, soft impute, mean imputation, and limit of detection.

We include a collection of workflows to format molecular phenotype data. These include workflows to separate phenotypes by chromosome, by user-provided regions, a workflow to subset bam files and a workflow to extract samples from phenotype files.


\paragraph*{C.  Covariate Data Preprocessing}


Our covariate preprocessing steps merge genotypic principal components and fixed covariate files into one file for downstream QTL analysis.

We provide two different procedures for hidden factor analysis from omics data in our pipeline. The first is the \href{https://github.com/PMBio/peer/wiki/Tutorial}{Probabilistic Estimation of Expression Residuals (PEER) method}, a method also used for GTEx eQTL data analysis. The second, and the one use for our main analyses, is a PCA based approach with automatic determination of the number of factors to use. This is mainly inspired by a recent benchmark from Jessica Li's group \href{https://doi.org/10.1186/s13059-022-02761-4}{[cf. Zhou et al., Genome Biology, 2022}]. Please note that additional considerations should be taken for single-cell eQTL analysis as investigated by \href{https://doi.org/10.1186/s13059-023-02873-5}{[cf. Xue et al., Genome Biology, 2023}].

\subsubsection*{QTL Association Testing (Step 4)}

\paragraph*{A.  QTL Association Analysis}


We perform QTL association testing using TensorQTL \href{https://doi.org/10.1186/s13059-019-1836-7}{[cf. Taylor-Weiner et al (2019)}].

\subsubsection*{Multivariate Mixture Model (Step 5)}

\paragraph*{A.   Mixture Multivariate Distribution Estimate}


\subsubsection*{Multiomics Regression Models (Step 6)}

\paragraph*{A.  Integrative Analysis with High-Dimensional Regression}


Our pipeline is capable of performing univariate fine-mapping with SuSiE with TWAS weights. The TWAS portion makes use of TWAS weights, linkage disequilibrium data and GWAS summary statistics. Preset variants used are taken from the linkage disequilibrium data and used only for TWAS. TWAS cross validation tell us which of the four methods (enet, lasso, mrash, SuSiE) are best to use. By default, we limit to under 5000 variants for cross validation. In cross validation, the data is split into five parts. Training is done on four parts, and prediction is done on the fifth. Linear regression is used to assess the results and get r squared and pvalues.



Fine mapping with SuSiE follows the formulat y=xb+e where x has many highly correlated variables due to linkage disequilibrium. Therefore, true effects (b), are very sparse. The SuSiE wrapper looks for five independent signals in each region to increase convergence speed. However, if five signals are found, then the the upper limit is increased. SuSiE does not allow for the inclusion of covariates. Therefore, covariates are regressed in.

Multi gene fine-mapping and TWAS may also be conducted with our pipeline. This considers multiple genes jointly within specific TAD windows.



This step is similar to the multivariate fine-mapping with two main differences. 1) TAD windows with multiple genes need to be defined. The `--pheno\_id\_map\_file` parameter is used for this. 2) To speed things up, the genes are filtered out if they don't have a univariate fine mapped region. Genes may also be filtered out if they do have a univariate fine-mapped signal, but the signal is nowhere close to that of other genes.  The `--skip-analysis-pip-cutoff` parameter is used for this.




Univariate fine-mapping for functional (epigenomic) data is conducted with fSuSiE. This is similar to the normal univariate fine-mapping, with the main difference being the use of epigonmic data.


Last, we include an option to conduct fine-mapping with SuSiE Regression using Summary Statistics (RSS) model and TWAS.


This notebook performs various advanced statistical analysis on multiple xQTL in a given region. Current procedures implemented include:



1. Univariate analysis

- SuSiE

- Univeriate TWAS weights: LASSO, Elastic net, mr.ash, SuSiE, Bayes alphabet soup

- Cross validation of TWAS methods (optional but highly recommended if TWAS weights are computed)

2. Functional data (epigenomic xQTL) analysis

- fSuSiE

3. Multivariate analysis

- mvSuSiE

- Multivariate TWAS weights: mvSuSiE and mr.mash

\subsubsection*{GWAS Integration (Step 7)}

\paragraph*{A.  xQTL-GWAS pairwise enrichment and colocalization}

\paragraph*{B.  TWAS, cTWAS and MR}


\paragraph*{C.  Multi-trait colocalization using ColocBoost}

\subsubsection*{Enrichment and Validation (Step 8)}

\paragraph*{A.  Chromosome-Specific Enrichment Analysis of Annotations Using Block Jackknife}


\section*{Materials}




\subsection*{Software}




\subsection*{Hardware}




\section*{Procedure}





\subsection*{1. Reference data}


\subsubsection*{Reference Data}
Timing ~4 hours

\paragraph*{i. Download Reference Data}


\noindent
\begin{lstlisting}[language=Python]

sos run pipeline/reference_data_preparation.ipynb download_hg_reference --cwd reference_data
sos run pipeline/reference_data_preparation.ipynb download_gene_annotation --cwd reference_data
sos run pipeline/reference_data_preparation.ipynb download_ercc_reference --cwd reference_data
sos run pipeline/reference_data_preparation.ipynb download_dbsnp --cwd reference_data

\end{lstlisting}




\paragraph*{ii. Format Reference Data}


\noindent
\begin{lstlisting}[language=Python]

sos run pipeline/reference_data_preparation.ipynb hg_reference \
    --cwd reference_data \
    --ercc-reference reference_data/ERCC92.fa \
    --hg-reference reference_data/GRCh38_full_analysis_set_plus_decoy_hla.fa

\end{lstlisting}




\paragraph*{iii. Format Gene Feature Data}


\noindent
\begin{lstlisting}[language=Python]

sos run pipeline/reference_data_preparation.ipynb gene_annotation \
    --cwd reference_data \
    --ercc-gtf reference_data/ERCC92.gtf \
    --hg-gtf reference_data/Homo_sapiens.GRCh38.103.chr.gtf \
    --hg-reference reference_data/GRCh38_full_analysis_set_plus_decoy_hla.noALT_noHLA_noDecoy.fasta \
    --stranded

\end{lstlisting}




\paragraph*{iv. Generate STAR Index}


\noindent
\begin{lstlisting}[language=Python]

sos run pipeline/reference_data_preparation.ipynb STAR_index \
    --cwd reference_data \
    --hg-reference reference_data/GRCh38_full_analysis_set_plus_decoy_hla.noALT_noHLA_noDecoy_ERCC.fasta

\end{lstlisting}




\paragraph*{v. Generate RSEM Index}


\noindent
\begin{lstlisting}[language=Python]

sos run pipeline/reference_data_preparation.ipynb RSEM_index \
    --cwd reference_data \
    --hg-reference reference_data/GRCh38_full_analysis_set_plus_decoy_hla.noALT_noHLA_noDecoy_ERCC.fasta \
    --hg-gtf reference_data/Homo_sapiens.GRCh38.103.chr.reformatted.ERCC.gtf

\end{lstlisting}




\paragraph*{vi. Generate RefFlat Annotation for Picard}


\noindent
\begin{lstlisting}[language=Python]

sos run pipeline/reference_data_preparation.ipynb RefFlat_generation \
    --cwd reference_data \
    --hg-gtf reference_data/Homo_sapiens.GRCh38.103.chr.reformatted.ERCC.gtf

\end{lstlisting}




\paragraph*{vii. Generate SUPPA Annotation for Psichomics}


\noindent
\begin{lstlisting}[language=Python]

sos run pipeline/reference_data_preparation.ipynb SUPPA_annotation \
    --cwd reference_data \
    --hg_gtf reference_data/Homo_sapiens.GRCh38.103.chr.reformatted.ERCC.gtf

\end{lstlisting}




\paragraph*{viii. Extract rsIDs for known variants}


\noindent
\begin{lstlisting}[language=Python]

sos run pipeline/VCF_QC.ipynb dbsnp_annotate \
    --genoFile reference_data/00-All.vcf.gz

\end{lstlisting}




\subsection*{2. Molecular Phenotypes}


\subsubsection*{RNA-seq expression}
Timing <3.5 hours

\paragraph*{i. Perform data quality summary via `fastqc`}


\noindent
\begin{lstlisting}[language=Python]

sos run pipeline/RNA_calling.ipynb fastqc \
    --cwd output/rnaseq/fastqc \
    --sample-list data/fastq.list.txt \
    --data-dir data/fastq

\end{lstlisting}




\paragraph*{ii. Cut adaptor (Optional)}


\noindent
\begin{lstlisting}[language=Python]

sos run pipeline/RNA_calling.ipynb fastp_trim_adaptor \
    --cwd output/rnaseq --sample-list data/fastq.list.txt \
    --data-dir data/fastq --STAR-index reference_data/STAR_Index/ \
    --gtf reference_data/Homo_sapiens.GRCh38.103.chr.reformatted.ERCC.gtf \
    --reference-fasta reference_data/GRCh38_full_analysis_set_plus_decoy_hla.noALT_noHLA_noDecoy_ERCC.fasta \
    --ref-flat reference_data/Homo_sapiens.GRCh38.103.chr.reformatted.ERCC.ref.flat

\end{lstlisting}




\paragraph*{iii. Read alignment via STAR and QC via Picard}


\noindent
\begin{lstlisting}[language=Python]

sos run pipeline/RNA_calling.ipynb STAR_align \
    --cwd output/rnaseq/bam --sample-list data/fastq.list.txt \
    --data-dir data/fastq --STAR-index reference_data/STAR_Index/ \
    --gtf reference_data/Homo_sapiens.GRCh38.103.chr.reformatted.ERCC.gtf \
    --reference-fasta reference_data/GRCh38_full_analysis_set_plus_decoy_hla.noALT_noHLA_noDecoy_ERCC.fasta \
    --ref-flat reference_data/Homo_sapiens.GRCh38.103.chr.reformatted.ERCC.ref.flat \
    --chimSegmentMin 0 \
    -J 50 --mem 200G --numThreads 8

\end{lstlisting}




\paragraph*{iv. Call gene-level RNA expression via rnaseqc}


\noindent
\begin{lstlisting}[language=Python]

sos run pipeline/RNA_calling.ipynb rnaseqc_call \
    --cwd data/bam \
    --sample-list data/fastq.list.txt \
    --data-dir data/fastq \
    --gtf reference_data/Homo_sapiens.GRCh38.103.chr.reformatted.collapse_only.gene.gtf \
    --reference-fasta reference_data/GRCh38_full_analysis_set_plus_decoy_hla.noALT_noHLA_noDecoy_ERCC.fasta \
    --varVCFfile reference_data/ZOD14598_AD_GRM_WGS_2021-04-29_all.recalibrated_variants.leftnorm.filtered.AF.WASP.vcf \
    --bam_list data/bam/sample_bam_list.txt

\end{lstlisting}




\paragraph*{v. Call transcript level RNA expression via RSEM}


\noindent
\begin{lstlisting}[language=Python]

sos run pipeline/RNA_calling.ipynb rsem_call \
    --cwd data/bam \
    --sample-list data/fastq.list.txt \
    --data-dir data/fastq \
    --STAR-index reference_data/STAR_Index/ \
    --gtf reference_data/Homo_sapiens.GRCh38.103.chr.reformatted.ERCC.gtf \
    --reference-fasta reference_data/GRCh38_full_analysis_set_plus_decoy_hla.noALT_noHLA_noDecoy_ERCC.fasta \
    --ref-flat reference_data/Homo_sapiens.GRCh38.103.chr.reformatted.ERCC.ref.flat \
    --varVCFfile reference_data/ZOD14598_AD_GRM_WGS_2021-04-29_all.recalibrated_variants.leftnorm.filtered.AF.WASP.vcf \
    --bam_list data/bam/sample_bam_list.txt \
    --RSEM-index reference_data/RSEM_Index

\end{lstlisting}




\paragraph*{vi. Multi-sample RNA-seq QC}


\noindent
\begin{lstlisting}[language=Python]

sos run pipeline/bulk_expression_QC.ipynb qc \
    --cwd output/rnaseq \
    --tpm-gct data/rnaseq/bulk_rnaseq_tmp_matrix.bed \
    --counts-gct data/rnaseq/bulk_rnaseq_count_matrix.bed

\end{lstlisting}




\paragraph*{vii. Multi-sample read count normalization}


\noindent
\begin{lstlisting}[language=Python]

sos run pipeline/bulk_expression_normalization.ipynb normalize \
    --cwd output/rnaseq \
    --tpm-gct output/rnaseq/bulk_rnaseq_tmp_matrix.low_expression_filtered.outlier_removed.tpm.gct.gz \
    --counts-gct output/rnaseq/bulk_rnaseq_tmp_matrix.low_expression_filtered.outlier_removed.geneCount.gct.gz \
    --annotation-gtf reference_data/Homo_sapiens.GRCh38.103.chr.reformatted.collapse_only.gene.ERCC.gtf  \
    --count-threshold 1 --sample_participant_lookup data/rnaseq/sample_participant_lookup.txt

\end{lstlisting}




\subsubsection*{Alternative splicing from RNA-seq data}
Timing <2 hours

\paragraph*{i. Splicing Quantification with Leafcutter (intron usage ratio) or Psichomics (percent spliced in events)}


\noindent
\begin{lstlisting}[language=Python]

sos run pipeline/splicing_calling.ipynb leafcutter \
    --cwd output/leaf_cutter/ \
    --samples output/rnaseq/xqtl_protocol_data_bam_list 

sos run pipeline/splicing_calling.ipynb psichomics \
    --cwd output/psichomics/ \
    --samples output/rnaseq/xqtl_protocol_data_bam_list \
    --splicing_annotation hg38_suppa.rds

\end{lstlisting}




\paragraph*{ii. Splicing QC and Normalization}


\noindent
\begin{lstlisting}[language=Python]

sos run pipeline/splicing_normalization.ipynb leafcutter_norm \
    --cwd output/leaf_cutter/ \
    --ratios output/leaf_cutter/xqtl_protocol_data_bam_list_intron_usage_perind.counts.gz

sos run pipeline/splicing_normalization.ipynb psichomics_norm \
    --cwd psichomics_output \
    --ratios psichomics_output/psi_raw_data.tsv

\end{lstlisting}




\paragraph*{iii. Post Processing for TensorQTL}


\noindent
\begin{lstlisting}[language=Python]

sos run pipeline/gene_annotation.ipynb annotate_leafcutter_isoforms \
    --cwd output/leaf_cutter/ \
    --intron_count output/leaf_cutter/xqtl_protocol_data_bam_list_intron_usage_perind_numers.counts.gz \
    --phenoFile output/leaf_cutter/xqtl_protocol_data_bam_list_intron_usage_perind.counts.gz_raw_data.qqnorm.txt \
    --annotation-gtf reference_data/Homo_sapiens.GRCh38.103.chr.reformatted.collapse_only.gene.gtf \
    --sample_participant_lookup reference_data/sample_participant_lookup.rnaseq

sos run pipeline/code/data_preprocessing/phenotype/gene_annotation.ipynb annotate_psichomics_isoforms \
    --cwd psichomics_output \
    --phenoFile psichomics_output/psichomics_raw_data_bedded.qqnorm.txt \
    --annotation-gtf reference_data/Homo_sapiens.GRCh38.103.chr.reformated.ERCC.gene.gtf

\end{lstlisting}




\subsection*{3. Data Pre-processing}


\subsubsection*{Genotype data preprocessing}
Timing < X minutes

\paragraph*{i. QC for VCF files}


\noindent
\begin{lstlisting}[language=Python]

echo ./ZOD14598_AD_GRM_WGS_2021-04-29_chr1.recalibrated_variants.vcf.gz ./ZOD14598_AD_GRM_WGS_2021-04-29_chr2.recalibrated_variants.vcf.gz ./ZOD14598_AD_GRM_WGS_2021-04-29_chr3.recalibrated_variants.vcf.gz ./ZOD14598_AD_GRM_WGS_2021-04-29_chr4.recalibrated_variants.vcf.gz ./ZOD14598_AD_GRM_WGS_2021-04-29_chr5.recalibrated_variants.vcf.gz ./ZOD14598_AD_GRM_WGS_2021-04-29_chr6.recalibrated_variants.vcf.gz ./ZOD14598_AD_GRM_WGS_2021-04-29_chr7.recalibrated_variants.vcf.gz ./ZOD14598_AD_GRM_WGS_2021-04-29_chr8.recalibrated_variants.vcf.gz ./ZOD14598_AD_GRM_WGS_2021-04-29_chr9.recalibrated_variants.vcf.gz ./ZOD14598_AD_GRM_WGS_2021-04-29_chr10.recalibrated_variants.vcf.gz  ./ZOD14598_AD_GRM_WGS_2021-04-29_chr11.recalibrated_variants.vcf.gz ./ZOD14598_AD_GRM_WGS_2021-04-29_chr12.recalibrated_variants.vcf.gz ./ZOD14598_AD_GRM_WGS_2021-04-29_chr13.recalibrated_variants.vcf.gz ./ZOD14598_AD_GRM_WGS_2021-04-29_chr14.recalibrated_variants.vcf.gz ./ZOD14598_AD_GRM_WGS_2021-04-29_chr15.recalibrated_variants.vcf.gz ./ZOD14598_AD_GRM_WGS_2021-04-29_chr16.recalibrated_variants.vcf.gz ./ZOD14598_AD_GRM_WGS_2021-04-29_chr17.recalibrated_variants.vcf.gz ./ZOD14598_AD_GRM_WGS_2021-04-29_chr18.recalibrated_variants.vcf.gz ./ZOD14598_AD_GRM_WGS_2021-04-29_chr19.recalibrated_variants.vcf.gz ./ZOD14598_AD_GRM_WGS_2021-04-29_chr20.recalibrated_variants.vcf.gz ./ZOD14598_AD_GRM_WGS_2021-04-29_chr21.recalibrated_variants.vcf.gz ./ZOD14598_AD_GRM_WGS_2021-04-29_chr22.recalibrated_variants.vcf.gz \
    | tr ' ' '\n' > /path/to/ZOD14598_AD_GRM_WGS_2021-04-29_vcf_files.txt
# We only do this for autosomal variants
sos run pipeline/VCF_QC.ipynb qc \
    --genoFile /path/to/ZOD14598_AD_GRM_WGS_2021-04-29_vcf_files.txt \
    --dbsnp-variants /path/to/reference_data/00-All.add_chr.variants.gz \
    --reference-genome /path/to/reference_data/GRCh38_full_analysis_set_plus_decoy_hla.noALT_noHLA_noDecoy_ERCC.fasta \
    --cwd vcf_qc/ \
    -J 22 --mem 120G

\end{lstlisting}




\paragraph*{ii. Merge separated bed files into one}


\noindent
\begin{lstlisting}[language=Python]

sos run pipeline/genotype_formatting.ipynb vcf_to_plink \
    --genoFile `ls data/WGS/vcf/wgs.chr*.random.vcf.gz` \
    --cwd output/plink/ 

sos run pipeline/genotype_formatting.ipynb merge_plink \
    --genoFile `ls output/plink/wgs.chr*.random.bed` \
    --name wgs.merged \
    --cwd output/plink/

\end{lstlisting}




\paragraph*{iii. QC for PLINK files}


\noindent
\begin{lstlisting}[language=Python]

sos run pipeline/GWAS_QC.ipynb qc_no_prune \
   --cwd output/plink \
   --genoFile output/plink/wgs.merged.bed \
   --geno-filter 0.1 \
   --mind-filter 0.1 \
   --hwe-filter 1e-08 \
   --mac-filter 0

\end{lstlisting}




\paragraph*{iv. Genotype data partition by chromosome}


\noindent
\begin{lstlisting}[language=Python]

sos run pipeline/genotype_formatting.ipynb genotype_by_chrom \
    --genoFile output/plink/wgs.merged.plink_qc.bed \
    --cwd output/genotype_by_chrom \
    --chrom `cut -f 1 output/plink/wgs.merged.plink_qc.bim | uniq | sed "s/chr//g"`

\end{lstlisting}




\paragraph*{v. Sample match with genotype}


\noindent
\begin{lstlisting}[language=Python]

sos run pipeline/GWAS_QC.ipynb genotype_phenotype_sample_overlap \
        --cwd output/genotype/ \
        --genoFile output/plink/wgs.merged.plink_qc.fam  \
        --phenoFile output/rnaseq/bulk_rnaseq_tmp_matrix.low_expression_filtered.outlier_removed.tmm.expression.bed.bed.gz

\end{lstlisting}




\paragraph*{vi. Kinship}


\noindent
\begin{lstlisting}[language=Python]

sos run pipeline/GWAS_QC.ipynb king \
    --cwd output/genotype/kinship \
    --genoFile output/plink/wgs.merged.plink_qc.bed \
    --name wgs.merged.king \
    --keep-samples output/genotype/bulk_rnaseq_tmp_matrix.low_expression_filtered.outlier_removed.tmm.expression.bed.bed.sample_genotypes.txt

\end{lstlisting}




\paragraph*{vii. Prepare unrelated individuals data for PCA}


\noindent
\begin{lstlisting}[language=Python]

sos run pipeline/GWAS_QC.ipynb qc \
   --cwd output/genotype/ \
   --genoFile output/genotype/kinship/wgs.merged.plink_qc.wgs.merged.king.unrelated.bed \
   --mac-filter 5 

#If `No related individuals detected from *.kin0` occurs, there is no separate genotype data generated for unrelated individuals. In this case, we need to work from the original genotype data and must use `--keep-samples` to run `qc` to extract samples for PCA.** For example:

sos run pipeline/GWAS_QC.ipynb qc \
   --cwd output/genotype/ \
   --genoFile output/plink/wgs.merged.plink_qc.bed \
   --mac-filter 5

\end{lstlisting}




\paragraph*{viii. PCA on genotype}


\noindent
\begin{lstlisting}[language=Python]

sos run pipeline/PCA.ipynb flashpca \
   --cwd output/genotype/genotype_pca \
   --genoFile output/genotype/wgs.merged.plink_qc.plink_qc.prune.bed

\end{lstlisting}




\subsubsection*{Phenotype data preprocessing}
Timing < 12 minutes

\paragraph*{i. Phenotype Annotation}


\noindent
\begin{lstlisting}[language=Python]

sos run pipeline/gene_annotation.ipynb annotate_coord \
    --cwd output/rnaseq \
    --phenoFile output/rnaseq/bulk_rnaseq_tmp_matrix.low_expression_filtered.outlier_removed.tmm.expression.bed.gz \
    --coordinate-annotation reference_data/Homo_sapiens.GRCh38.103.chr.reformatted.collapse_only.gene.ERCC.gtf \
    --phenotype-id-column gene_id

\end{lstlisting}




\paragraph*{ii. Missing Value Imputation}


\noindent
\begin{lstlisting}[language=Python]

sos run pipeline/phenotype_imputation.ipynb gEBMF \
    --phenoFile data/protocol_example.protein.bed.gz \
    --cwd output/phenotype/impute_gebmf \
    --no-qc-prior-to-impute

\end{lstlisting}




\paragraph*{iii. Partition by Chromosome}


\noindent
\begin{lstlisting}[language=Python]

sos run pipeline/phenotype_formatting.ipynb phenotype_by_chrom \
    --cwd output/phenotype/phenotype_by_chrom \
    --phenoFile output/rnaseq/bulk_rnaseq_tmp_matrix.low_expression_filtered.outlier_removed.tmm.expression.bed.bed.gz \
    --name bulk_rnaseq \
    --chrom `for i in {1..22}; do echo chr$i; done`

\end{lstlisting}




\subsubsection*{Covariate Data Preprocessing}
Timing < 3 minutes

\paragraph*{i. Merge Covariates and Genotype PCs}


\noindent
\begin{lstlisting}[language=Python]

sos run pipeline/covariate_formatting.ipynb merge_genotype_pc \
    --cwd output/covariate/ \
    --pcaFile output/genotype/genotype_pca/wgs.merged.plink_qc.plink_qc.prune.pca.rds \
    --covFile data/covariate/covariates.tsv \
    --tol-cov 0.4 \
    --k `awk '$3 < 0.8' output/genotype/genotype_pca/wgs.merged.plink_qc.plink_qc.prune.pca.scree.txt | tail -1 | cut -f 1 `

\end{lstlisting}




\paragraph*{ii. Compute Residual on Merged Covariates and Perform Hidden Factor Analysis}


\noindent
\begin{lstlisting}[language=Python]

sos run pipeline/covariate_hidden_factor.ipynb Marchenko_PC \
   --cwd output/covariate \
   --phenoFile output/rnaseq/bulk_rnaseq_tmp_matrix.low_expression_filtered.outlier_removed.tmm.expression.bed.bed.gz  \
   --covFile output/covariate/covariates.wgs.merged.plink_qc.plink_qc.prune.pca.gz \
   --mean-impute-missing

\end{lstlisting}




\subsection*{4. QTL Association Testing}


\subsubsection*{QTL Association Analysis}
Timing < X minutes

\paragraph*{i. Cis TensorQTL Command}


\noindent
\begin{lstlisting}[language=Python]

sos run pipeline/TensorQTL.ipynb cis \
    --genotype-file output/genotype_by_chrom/wgs.merged.plink_qc.genotype_by_chrom_files.txt \
    --phenotype-file output/phenotype/phenotype_by_chrom/bulk_rnaseq.phenotype_by_chrom_files.txt \
    --covariate-file output/covariate/covariates.wgs.merged.plink_qc.plink_qc.prune.pca.gz \
    --customized-cis-windows reference_data/TAD/TADB_enhanced_cis.bed \
    --cwd output/tensorqtl_cis/ \
    --MAC 5

\end{lstlisting}




\paragraph*{ii. Trans TensorQTL Command}


\noindent
\begin{lstlisting}[language=Python]

sos run pipeline/TensorQTL.ipynb trans \
    --genotype-file data/wgs.merged.plink_qc.genotype_trans_files.txt \
    --phenotype-file output/phenotype/phenotype_by_chrom_for_trans/bulk_rnaseq.phenotype_by_chrom_files.txt \
    --region-list data/combined_AD_genes.csv \
    --region-list-phenotype-column 4 \
    --covariate-file output/covariate/bulk_rnaseq_tmp_matrix.low_expression_filtered.outlier_removed.tmm.expression.covariates.wgs.merged.plink_qc.plink_qc.prune.pca.Marchenko_PC.gz \
    --cwd output/tensorqtl_trans/ \
    --MAC 5

\end{lstlisting}




\paragraph*{iii. Interaction TensorQTL Command}


\noindent
\begin{lstlisting}[language=Python]

sos run pipeline/TensorQTL.ipynb cis \
    --genotype-file output/genotype_by_chrom/wgs.merged.plink_qc.genotype_by_chrom_files.txt \
    --phenotype-file output/phenotype/phenotype_by_chrom/bulk_rnaseq.phenotype_by_chrom_files.txt \
    --covariate-file output/covariate/covariates.wgs.merged.plink_qc.plink_qc.prune.pca.gz \
    --customized-cis-windows reference_data/TAD/TADB_enhanced_cis.bed \
    --cwd output/tensorqtl_int/ \
    --no-permutation \
    --maf-threshold 0.05 \
    --interaction sex \
    -j 22

\end{lstlisting}




\subsection*{5. Multivariate Mixture Model}


\subsubsection*{Mixture Multivariate Distribution Estimate}
Timing ~X minutes

\paragraph*{i. Compute MASH prior}


\noindent
\begin{lstlisting}[language=Python]

sos run pipeline/mixture_prior.ipynb ed_bovy \
    --output_prefix MWE_ed_bovy \
    --data data/multivariate_mixture/MWE.rds \
    --cwd output/multivariate_mixture --vhat mle

\end{lstlisting}




\paragraph*{ii. MASH fit}


\noindent
\begin{lstlisting}[language=Python]

sos run pipeline/mash_fit.ipynb mash \
    --output-prefix MWE_ed_bovy_posterior \
    --data data/multivariate_mixture/MWE.rds \
    --vhat-data output/multivariate_mixture/MWE_ed_bovy.EE.V_simple.rds \
    --prior-data output/multivariate_mixture/MWE_ed_bovy.EE.prior.rds \
    --compute-posterior \
    --cwd output/multivariate_mixture

\end{lstlisting}




\paragraph*{iii. Generate Plots}


\noindent
\begin{lstlisting}[language=Python]

sos run pipeline/mixture_prior.ipynb plot_U \
    --output-prefix protocol_example.mixture_plots  \
    --data output/multivariate_mixture/MWE_ed_bovy.EE.prior.rds \
    --cwd output/multivariate_mixture

\end{lstlisting}




\subsection*{6. Multiomics Regression Models}


\subsubsection*{Integrative Analysis with High-Dimensional Regression}
Timing < X minutes

\paragraph*{i. Univariate Fine-Mapping and TWAS with SuSiE}


\noindent
\begin{lstlisting}[language=Python]

sos run pipeline/mnm_regression.ipynb susie_twas \
    --name test_susie_twas \
    --genoFile output/genotype_by_chrom/wgs.merged.plink_qc.1.bed \
    --phenoFile output/phenotype/phenotype_by_chrom_for_cis/bulk_rnaseq.phenotype_by_chrom_files.region_list.txt \
    --covFile output/covariate/bulk_rnaseq_tmp_matrix.low_expression_filtered.outlier_removed.tmm.expression.covariates.wgs.merged.plink_qc.plink_qc.prune.pca.Marchenko_PC.gz \
    --customized-association-windows reference_data/TAD/TADB_enhanced_cis.bed \
    --phenotype-names test_pheno \
    --max-cv-variants 5000 --ld_reference_meta_file data/ld_meta_file_with_bim.tsv \
    --region-name ENSG00000049246 ENSG00000054116 ENSG00000116678 \
    --save-data \
    --cwd output/mnm_regression/susie_twas

\end{lstlisting}




\paragraph*{ii. Multivariate Fine-Mapping for multiple genes}


\noindent
\begin{lstlisting}[language=Python]

sos run pipeline/mnm_regression.ipynb mnm_genes \
    --name ROSMAP_Ast_mega_eQTL \
    --genoFile data/mnm_genes/ROSMAP_NIA_WGS.leftnorm.bcftools_qc.plink_qc.11.bed \
    --phenoFile data/mnm_genes/snuc_pseudo_bulk.Ast.mega.normalized.log2cpm.region_list.txt \
    --covFile data/mnm_genes/snuc_pseudo_bulk.Ast.mega.normalized.log2cpm.rosmap_cov.ROSMAP_NIA_WGS.leftnorm.bcftools_qc.plink_qc.snuc_pseudo_bulk_mega.related.plink_qc.extracted.pca.projected.Marchenko_PC.gz \
    --customized-association-windows data/mnm_genes/extended_TADB.bed \
    --phenotype-names Ast_mega_eQTL \
    --max-cv-variants 5000 \
    --ld_reference_meta_file data/ld_meta_file_with_bim.tsv \
    --independent_variant_list data/mnm_genes/ld_pruned_variants.txt.gz \
    --fine_mapping_meta data/mnm_genes/combined_data_updated.tsv \
    --phenoIDFile data/mnm_genes/phenoIDFile_extended_TADB.bed \
    --region-name chr11_77324757_82556425 \
    --skip-analysis-pip-cutoff 0 \
    --maf 0.01 \
    --coverage 0.95 \
    --pheno_id_map_file data/mnm_genes/pheno_id_map_file.txt \
    --prior-canonical-matrices \
    --twas-cv-folds 0 \
    --trans-analysis \
    --cwd output/mnm_regression/mnm_genes -s build

\end{lstlisting}




\paragraph*{iii. Univariate Fine-Mapping of Functional (Epigenomic) Data with fSuSiE}


\noindent
\begin{lstlisting}[language=Python]

sos run pipeline/mnm_regression.ipynb fsusie \
    --cwd output/fsusie/ \
    --name test_fsusie \
    --genoFile output/genotype_by_chrom/wgs.merged.plink_qc.genotype_by_chrom_files.txt \
    --phenoFile output/phenotype/phenotype_by_chrom_for_cis/bulk_rnaseq.phenotype_by_chrom_files.region_list.txt \
    --covFile output/covariate/bulk_rnaseq_tpm_matrix.low_expression_filtered.outlier_removed.tmm.expression.covariates.wgs.merged.plink_qc.plink_qc.prune.pca.Marchenko_PC.gz \
    --numThreads 8 \
    --customized-association-windows reference_data/TAD/TADB_enhanced_cis.bed \
    --save-data \
    --region-name ENSG00000049246 ENSG00000054116 ENSG00000116678 ENSG00000073921 ENSG00000186891

\end{lstlisting}




\paragraph*{iv. Multivariate Fine-Mapping with mvSuSiE and mr.mash}


\noindent
\begin{lstlisting}[language=Python]

sos run pipeline/mnm_regression.ipynb mnm \
    --name test_mnm --cwd output/mnm \
    --genoFile output/genotype_by_chrom/wgs.merged.plink_qc.genotype_by_chrom_files.txt \
    --phenoFile output/phenotype/phenotype_by_chrom_for_cis/bulk_rnaseq.phenotype_by_chrom_files.region_list.txt \
    --covFile output/covariate/bulk_rnaseq_tpm_matrix.low_expression_filtered.outlier_removed.tmm.expression.covariates.wgs.merged.plink_qc.plink_qc.prune.pca.Marchenko_PC.gz \
    --customized-association-windows reference_data/TAD/TADB_enhanced_cis.bed \
    --region-name ENSG00000073921 --save-data --no-skip-twas-weights \
    --phenotype-names test_pheno \
    --mixture_prior output/multivariate_mixture/MWE_ed_bovy.EE.prior.rds \
    --max_cv_variants 5000 \
	--ld_reference_meta_file data/ld_meta_file.tsv

\end{lstlisting}




\paragraph*{v. Regression with Summary Statistics (RSS) Fine-Mapping and TWAS with SuSiE}


\noindent
\begin{lstlisting}[language=Python]

sos run pipeline/rss_analysis.ipynb univariate_rss \
    --ld-meta-data data/ld_meta_file_with_bim.tsv \
    --gwas-meta-data data/mnm_regression/gwas_meta_data.txt \
    --qc_method "rss_qc" --impute \
    --finemapping_method "susie_rss" \
    --cwd output/rss_analysis \
    --skip_analysis_pip_cutoff 0 \
    --skip_regions 6:25000000-35000000 \
    --region_name 22:49355984-50799822

\end{lstlisting}




\subsection*{7. GWAS Integration}


\subsubsection*{xQTL-GWAS pairwise enrichment and colocalization}
Timing: ~X min

\subsubsection*{i. Enrichment}


\noindent
\begin{lstlisting}[language=Python]

sos run pipeline/SuSiE_enloc.ipynb xqtl_gwas_enrichment    \
    --gwas-meta-data data/susie_enloc_data/demo_gwas.block_results_db.tsv \
    --xqtl-meta-data data/susie_enloc_data/demo_overlap.overlapped.gwas.tsv \
    --xqtl_finemapping_obj preset_variants_result susie_result_trimmed  \
    --xqtl_varname_obj preset_variants_result variant_names  \
    --gwas_finemapping_obj AD_Bellenguez_2022 RSS_QC_RAISS_imputed susie_result_trimmed \
    --gwas_varname_obj  AD_Bellenguez_2022 RSS_QC_RAISS_imputed variant_names \
    --xqtl_region_obj  region_info   grange \
    --qtl-path data/susie_enloc_data \
    --gwas_path data/susie_enloc_data \
    --context_meta data/susie_enloc_data/context_meta.tsv \
    --cwd output/xqtl_gwas_enrichment

\end{lstlisting}




\subsubsection*{ii. Coloc}
This would trigger enrichment automatically. Include the `--skip-enrich` to skip the enrichment part. Otherwise, the `*enrichment.rds` file from the previous step must be in the location of the `--cwd` directory.


\noindent
\begin{lstlisting}[language=Python]

sos run pipeline/SuSiE_enloc.ipynb susie_coloc \
    --gwas-meta-data data/susie_enloc_data/demo_gwas.block_results_db.tsv \
    --xqtl-meta-data data/susie_enloc_data/demo_overlap.overlapped.gwas.tsv \
    --xqtl_finemapping_obj preset_variants_result susie_result_trimmed \
    --xqtl_varname_obj preset_variants_result variant_names \
    --gwas_finemapping_obj AD_Bellenguez_2022 RSS_QC_RAISS_imputed susie_result_trimmed \
    --gwas_varname_obj  AD_Bellenguez_2022 RSS_QC_RAISS_imputed variant_names \
    --xqtl_region_obj  region_info grange \
    --qtl-path data/susie_enloc_data \
    --gwas_path data/susie_enloc_data \
    --context_meta data/susie_enloc_data/context_meta.tsv \
    --ld_meta_file_path /restricted/projectnb/xqtl/xqtl_protocol/scripts/pixi_scripts/ld_meta_file.tsv \
    --cwd output/susie_coloc

\end{lstlisting}




\subsubsection*{TWAS, cTWAS and MR}
Timing: ~X min

\paragraph*{iii. Run TWAS}


\noindent
\begin{lstlisting}[language=Python]

sos run pipeline/twas_ctwas.ipynb twas \
   --cwd output/twas --name test \
   --gwas_meta_data data/twas/gwas_meta_test.tsv \
   --ld_meta_data reference_data/ADSP_R4_EUR/ld_meta_file.tsv \
   --regions data/twas/EUR_LD_blocks.bed \
   --xqtl_meta_data data/twas/mwe_twas_pipeline_test_small.tsv \
   --xqtl_type_table data/twas/data_type_table.txt \
   --rsq_pval_cutoff 0.05 --rsq_cutoff 0.01

\end{lstlisting}




\paragraph*{iv. Run cTWAS}


\noindent
\begin{lstlisting}[language=Python]

sos run pipeline/twas_ctwas.ipynb ctwas \
   --cwd output/twas --name test \
   --gwas_meta_data data/twas/gwas_meta_test.tsv \
   --ld_meta_data data/ld_meta_file_with_bim.tsv \
   --xqtl_meta_data data/twas/mwe_twas_pipeline_test_small.tsv \
   --twas_weight_cutoff 0 \
   --chrom 11 \
   --regions data/twas/EUR_LD_blocks.bed \
   --region-name chr10_80126158_82231647 chr11_84267999_86714492

\end{lstlisting}




\subsubsection*{Multi-trait colocalization using ColocBoost}
Timing: ~X min

\paragraph*{v. Colocboost}


\noindent
\begin{lstlisting}[language=Python]

sos run pipeline/colocboost.ipynb colocboost \
    --name test_coloc_boost_xqtl_only  \
    --cwd output/colocboost \
    --genoFile output/plink/wgs.merged.bed \
    --phenoFile output/phenotype/phenotype_by_chrom_for_cis/bulk_rnaseq.phenotype_by_chrom_files.region_list.txt \
    --covFile output/covariate/bulk_rnaseq_tmp_matrix.low_expression_filtered.outlier_removed.tmm.expression.covariates.wgs.merged.plink_qc.plink_qc.prune.pca.Marchenko_PC.gz \
    --customized-association-windows reference_data/TAD/TADB_enhanced_cis.bed \
    --no-separate-gwas --xqtl-coloc \
    --region-name ENSG00000239945 \
    --phenotype-names trait_A

#It is also possible to analyze a selected list of regions using option `--region-list`. The last column of this file will be used for the list to analyze. Here for example use the same list of regions as we used for customized association-window:
sos run pipeline/colocboost.ipynb colcoboost  \
    --name protocol_example_protein  \
    --genoFile input/xqtl_association/protocol_example.genotype.chr21_22.bed   \
    --phenoFile output/phenotype/protocol_example.protein.region_list.txt \
                output/phenotype/protocol_example.protein.region_list.txt \
    --covFile output/covariate/protocol_example.protein.protocol_example.samples.protocol_example.genotype.chr21_22.pQTL.plink_qc.prune.pca.Marchenko_PC.gz \
              output/covariate/protocol_example.protein.protocol_example.samples.protocol_example.genotype.chr21_22.pQTL.plink_qc.prune.pca.Marchenko_PC.gz  \
    --customized-association-windows input/xqtl_association/protocol_example.protein.enhanced_cis_chr21_chr22.bed \
    --no-separate-gwas --xqtl-coloc \
    --region-list xqtl_association/protocol_example.protein.enhanced_cis_chr21_chr22.bed \
    --phenotype-names trait_A trait_B

\end{lstlisting}




\paragraph*{vi. Colocboost with GWAS}


\noindent
\begin{lstlisting}[language=Python]

sos run pipeline/colocboost.ipynb colocboost \
    --name colocboost_gwas  \
    --cwd output/colocboost_gwas \
    --genoFile output/plink/wgs.merged.bed \
    --phenoFile output/phenotype/phenotype_by_chrom_for_cis/bulk_rnaseq.phenotype_by_chrom_files.region_list.txt \
    --covFile output/covariate/bulk_rnaseq_tmp_matrix.low_expression_filtered.outlier_removed.tmm.expression.covariates.wgs.merged.plink_qc.plink_qc.prune.pca.Marchenko_PC.gz \
    --customized-association-windows reference_data/TAD/TADB_enhanced_cis.bed \
    --ld-meta-data data/ld_meta_file.tsv \
    --gwas-meta-data data/colocboost/gwas_meta.txt \
    --separate-gwas --xqtl-coloc \
    --region-name ENSG00000239945 \
    --phenotype-names trait_A

\end{lstlisting}




\subsection*{8. Enrichment and Validation}


\subsubsection*{Chromosome-Specific Enrichment Analysis of Annotations Using Block Jackknife}
Timing: ~X min

\paragraph*{i. Enrichment}


\noindent
\begin{lstlisting}[language=Python]

sos run pipeline/eoo_enrichment.ipynb enrichment \
    --significant_variants_path data/eoo_enrichment/colocboost_binary_vcp0.1_hg38_annotation.tsv.gz \
    --baseline_anno_path data/eoo_enrichment/colocboost_binary_vcp0.1_hg38_annotation_data.tsv \
    --name enrichment_results \
    --cwd output/eoo_enrichment

\end{lstlisting}





\section*{Timing}




\begin{tabular}{|l|l|}
\hline
Step & Time \\
\hline
Reference data & X minutes \\
Molecular Phenotypes & X minutes \\
Data Pre-processing & X minutes \\
QTL Association Testing & X minutes \\
Multivariate Mixture Model & X minutes \\
Multiomics Regression Models & X minutes \\
GWAS Integration & X minutes \\
Enrichment and Validation & X minutes \\
\hline
\end{tabular}


\section*{Troubleshooting}





\section*{Anticipated Results}





\subsubsection*{Reference Data}

Our pipeline uses the following reference data for RNA-seq expression quantification:


\subsubsection*{RNA-seq expression}

The final output contained QCed and normalized expression data in a bed.gz file. This file is ready for use in TensorQTL.

\subsubsection*{Alternative splicing from RNA-seq data}

The final output contains the QCed and normalized splicing data from leafcutter and psichomics.

\subsubsection*{Genotype data preprocessing}

Genotype preprocessing will produce cleaned genotype files, and genetic principal components.

\subsubsection*{Phenotype data preprocessing}

Phenotype preprocessing should result in a phenotype file formatted and ready for use in TensorQTL.

\subsubsection*{Covariate Data Preprocessing}

Processed covariate data includes a file with covariates and hidden factors for use in TensorQTL.

\subsubsection*{QTL Association Analysis}

TensorQTL will produce empirical and standardized cis/trans results.

\subsubsection*{Mixture Multivariate Distribution Estimate}

i. Compute MASH prior


\subsubsection*{Integrative Analysis with High-Dimensional Regression}

Univariate finemapping will produce a file containing results for the top hits and a file containing twas weights produced by susie. Multigene finemapping with mvSuSiE will produce a file for each gene and region containing results for the top hits and a file containing twas weights produced by susie. Univariate finemapping for functional data with fSuSiE will produce a file containing results for the top hits and a file containing residuals from SuSiE. Multivariate finemapping will produce a file containing results for the top hits for each gene and a file containing twas weights produced by susie. Summary statistics fine-mapping produces a results file for each region and gwas of interest.

\subsubsection*{xQTL-GWAS pairwise enrichment and colocalization}

\subsubsection*{TWAS, cTWAS and MR}


\subsubsection*{Multi-trait colocalization using ColocBoost}

\subsubsection*{Chromosome-Specific Enrichment Analysis of Annotations Using Block Jackknife}


\section*{Figures}





\section*{Tables}





\section*{Supplementary Information}





\section*{Author Contributions Statements}




\section*{Acknowledgements}




\section*{Competing Interests}




\section*{References}
\begin{enumerate}
  \item McArthur et al. 2021. \url{https://doi.org/10.1016%2Fj.ajhg.2021.01.001}
  \item Schmitt et al. 2017. \url{https://doi.org/10.1016%2Fj.celrep.2016.10.061}
  \item Signal et al. 2022. \url{https://doi.org/10.1186/s12859-022-04572-7}
  \item DeLuca et al. 2012. \url{https://doi.org/10.1093/bioinformatics/bts196}
  \item Wang et al. 2008. \url{https://doi.org/10.1038/nature07509}
  \item Park et al. 2018. \url{https://doi.org/10.1016/j.ajhg.2017.11.002}
  \item Li et al. 2018. \url{https://doi.org/10.1038/s41588-017-0004-9}
  \item Agostinho et al. 2019. \url{https://doi.org/10.1093/nar/gky888}
  \item Carson et al. 2014. \url{https://doi.org/10.1186/1471-2105-15-125}
  \item Lek et al. 2016. \url{https://doi.org/10.1038/nature19057}
  \item Szustakowski et al. 2021. \url{https://doi.org/10.1038/s41588-021-00885-0}
  \item Qi et al. 2023. \url{https://doi.org/10.1101/2023.11.29.23299181}
  \item Zhou et al. 2022. \url{https://doi.org/10.1186/s13059-022-02761-4}
  \item Xue et al. 2023. \url{https://doi.org/10.1186/s13059-023-02873-5}
  \item Taylor-Weiner et al. 2019. \url{https://doi.org/10.1186/s13059-019-1836-7}
\end{enumerate}

\section*{Keywords}





\end{document}